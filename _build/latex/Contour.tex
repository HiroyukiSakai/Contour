% Generated by Sphinx.
\def\sphinxdocclass{report}
\documentclass[letterpaper,10pt,english]{sphinxmanual}
\usepackage[utf8]{inputenc}
\DeclareUnicodeCharacter{00A0}{\nobreakspace}
\usepackage{cmap}
\usepackage[T1]{fontenc}
\usepackage{babel}
\usepackage{times}
\usepackage[Bjarne]{fncychap}
\usepackage{longtable}
\usepackage{sphinx}
\usepackage{multirow}


\title{Contour Documentation}
\date{February 09, 2014}
\release{1.0.1}
\author{Hiroyuki Sakai}
\newcommand{\sphinxlogo}{}
\renewcommand{\releasename}{Release}
\makeindex

\makeatletter
\def\PYG@reset{\let\PYG@it=\relax \let\PYG@bf=\relax%
    \let\PYG@ul=\relax \let\PYG@tc=\relax%
    \let\PYG@bc=\relax \let\PYG@ff=\relax}
\def\PYG@tok#1{\csname PYG@tok@#1\endcsname}
\def\PYG@toks#1+{\ifx\relax#1\empty\else%
    \PYG@tok{#1}\expandafter\PYG@toks\fi}
\def\PYG@do#1{\PYG@bc{\PYG@tc{\PYG@ul{%
    \PYG@it{\PYG@bf{\PYG@ff{#1}}}}}}}
\def\PYG#1#2{\PYG@reset\PYG@toks#1+\relax+\PYG@do{#2}}

\expandafter\def\csname PYG@tok@gd\endcsname{\def\PYG@tc##1{\textcolor[rgb]{0.63,0.00,0.00}{##1}}}
\expandafter\def\csname PYG@tok@gu\endcsname{\let\PYG@bf=\textbf\def\PYG@tc##1{\textcolor[rgb]{0.50,0.00,0.50}{##1}}}
\expandafter\def\csname PYG@tok@gt\endcsname{\def\PYG@tc##1{\textcolor[rgb]{0.00,0.27,0.87}{##1}}}
\expandafter\def\csname PYG@tok@gs\endcsname{\let\PYG@bf=\textbf}
\expandafter\def\csname PYG@tok@gr\endcsname{\def\PYG@tc##1{\textcolor[rgb]{1.00,0.00,0.00}{##1}}}
\expandafter\def\csname PYG@tok@cm\endcsname{\let\PYG@it=\textit\def\PYG@tc##1{\textcolor[rgb]{0.25,0.50,0.56}{##1}}}
\expandafter\def\csname PYG@tok@vg\endcsname{\def\PYG@tc##1{\textcolor[rgb]{0.73,0.38,0.84}{##1}}}
\expandafter\def\csname PYG@tok@m\endcsname{\def\PYG@tc##1{\textcolor[rgb]{0.13,0.50,0.31}{##1}}}
\expandafter\def\csname PYG@tok@mh\endcsname{\def\PYG@tc##1{\textcolor[rgb]{0.13,0.50,0.31}{##1}}}
\expandafter\def\csname PYG@tok@cs\endcsname{\def\PYG@tc##1{\textcolor[rgb]{0.25,0.50,0.56}{##1}}\def\PYG@bc##1{\setlength{\fboxsep}{0pt}\colorbox[rgb]{1.00,0.94,0.94}{\strut ##1}}}
\expandafter\def\csname PYG@tok@ge\endcsname{\let\PYG@it=\textit}
\expandafter\def\csname PYG@tok@vc\endcsname{\def\PYG@tc##1{\textcolor[rgb]{0.73,0.38,0.84}{##1}}}
\expandafter\def\csname PYG@tok@il\endcsname{\def\PYG@tc##1{\textcolor[rgb]{0.13,0.50,0.31}{##1}}}
\expandafter\def\csname PYG@tok@go\endcsname{\def\PYG@tc##1{\textcolor[rgb]{0.20,0.20,0.20}{##1}}}
\expandafter\def\csname PYG@tok@cp\endcsname{\def\PYG@tc##1{\textcolor[rgb]{0.00,0.44,0.13}{##1}}}
\expandafter\def\csname PYG@tok@gi\endcsname{\def\PYG@tc##1{\textcolor[rgb]{0.00,0.63,0.00}{##1}}}
\expandafter\def\csname PYG@tok@gh\endcsname{\let\PYG@bf=\textbf\def\PYG@tc##1{\textcolor[rgb]{0.00,0.00,0.50}{##1}}}
\expandafter\def\csname PYG@tok@ni\endcsname{\let\PYG@bf=\textbf\def\PYG@tc##1{\textcolor[rgb]{0.84,0.33,0.22}{##1}}}
\expandafter\def\csname PYG@tok@nl\endcsname{\let\PYG@bf=\textbf\def\PYG@tc##1{\textcolor[rgb]{0.00,0.13,0.44}{##1}}}
\expandafter\def\csname PYG@tok@nn\endcsname{\let\PYG@bf=\textbf\def\PYG@tc##1{\textcolor[rgb]{0.05,0.52,0.71}{##1}}}
\expandafter\def\csname PYG@tok@no\endcsname{\def\PYG@tc##1{\textcolor[rgb]{0.38,0.68,0.84}{##1}}}
\expandafter\def\csname PYG@tok@na\endcsname{\def\PYG@tc##1{\textcolor[rgb]{0.25,0.44,0.63}{##1}}}
\expandafter\def\csname PYG@tok@nb\endcsname{\def\PYG@tc##1{\textcolor[rgb]{0.00,0.44,0.13}{##1}}}
\expandafter\def\csname PYG@tok@nc\endcsname{\let\PYG@bf=\textbf\def\PYG@tc##1{\textcolor[rgb]{0.05,0.52,0.71}{##1}}}
\expandafter\def\csname PYG@tok@nd\endcsname{\let\PYG@bf=\textbf\def\PYG@tc##1{\textcolor[rgb]{0.33,0.33,0.33}{##1}}}
\expandafter\def\csname PYG@tok@ne\endcsname{\def\PYG@tc##1{\textcolor[rgb]{0.00,0.44,0.13}{##1}}}
\expandafter\def\csname PYG@tok@nf\endcsname{\def\PYG@tc##1{\textcolor[rgb]{0.02,0.16,0.49}{##1}}}
\expandafter\def\csname PYG@tok@si\endcsname{\let\PYG@it=\textit\def\PYG@tc##1{\textcolor[rgb]{0.44,0.63,0.82}{##1}}}
\expandafter\def\csname PYG@tok@s2\endcsname{\def\PYG@tc##1{\textcolor[rgb]{0.25,0.44,0.63}{##1}}}
\expandafter\def\csname PYG@tok@vi\endcsname{\def\PYG@tc##1{\textcolor[rgb]{0.73,0.38,0.84}{##1}}}
\expandafter\def\csname PYG@tok@nt\endcsname{\let\PYG@bf=\textbf\def\PYG@tc##1{\textcolor[rgb]{0.02,0.16,0.45}{##1}}}
\expandafter\def\csname PYG@tok@nv\endcsname{\def\PYG@tc##1{\textcolor[rgb]{0.73,0.38,0.84}{##1}}}
\expandafter\def\csname PYG@tok@s1\endcsname{\def\PYG@tc##1{\textcolor[rgb]{0.25,0.44,0.63}{##1}}}
\expandafter\def\csname PYG@tok@gp\endcsname{\let\PYG@bf=\textbf\def\PYG@tc##1{\textcolor[rgb]{0.78,0.36,0.04}{##1}}}
\expandafter\def\csname PYG@tok@sh\endcsname{\def\PYG@tc##1{\textcolor[rgb]{0.25,0.44,0.63}{##1}}}
\expandafter\def\csname PYG@tok@ow\endcsname{\let\PYG@bf=\textbf\def\PYG@tc##1{\textcolor[rgb]{0.00,0.44,0.13}{##1}}}
\expandafter\def\csname PYG@tok@sx\endcsname{\def\PYG@tc##1{\textcolor[rgb]{0.78,0.36,0.04}{##1}}}
\expandafter\def\csname PYG@tok@bp\endcsname{\def\PYG@tc##1{\textcolor[rgb]{0.00,0.44,0.13}{##1}}}
\expandafter\def\csname PYG@tok@c1\endcsname{\let\PYG@it=\textit\def\PYG@tc##1{\textcolor[rgb]{0.25,0.50,0.56}{##1}}}
\expandafter\def\csname PYG@tok@kc\endcsname{\let\PYG@bf=\textbf\def\PYG@tc##1{\textcolor[rgb]{0.00,0.44,0.13}{##1}}}
\expandafter\def\csname PYG@tok@c\endcsname{\let\PYG@it=\textit\def\PYG@tc##1{\textcolor[rgb]{0.25,0.50,0.56}{##1}}}
\expandafter\def\csname PYG@tok@mf\endcsname{\def\PYG@tc##1{\textcolor[rgb]{0.13,0.50,0.31}{##1}}}
\expandafter\def\csname PYG@tok@err\endcsname{\def\PYG@bc##1{\setlength{\fboxsep}{0pt}\fcolorbox[rgb]{1.00,0.00,0.00}{1,1,1}{\strut ##1}}}
\expandafter\def\csname PYG@tok@kd\endcsname{\let\PYG@bf=\textbf\def\PYG@tc##1{\textcolor[rgb]{0.00,0.44,0.13}{##1}}}
\expandafter\def\csname PYG@tok@ss\endcsname{\def\PYG@tc##1{\textcolor[rgb]{0.32,0.47,0.09}{##1}}}
\expandafter\def\csname PYG@tok@sr\endcsname{\def\PYG@tc##1{\textcolor[rgb]{0.14,0.33,0.53}{##1}}}
\expandafter\def\csname PYG@tok@mo\endcsname{\def\PYG@tc##1{\textcolor[rgb]{0.13,0.50,0.31}{##1}}}
\expandafter\def\csname PYG@tok@mi\endcsname{\def\PYG@tc##1{\textcolor[rgb]{0.13,0.50,0.31}{##1}}}
\expandafter\def\csname PYG@tok@kn\endcsname{\let\PYG@bf=\textbf\def\PYG@tc##1{\textcolor[rgb]{0.00,0.44,0.13}{##1}}}
\expandafter\def\csname PYG@tok@o\endcsname{\def\PYG@tc##1{\textcolor[rgb]{0.40,0.40,0.40}{##1}}}
\expandafter\def\csname PYG@tok@kr\endcsname{\let\PYG@bf=\textbf\def\PYG@tc##1{\textcolor[rgb]{0.00,0.44,0.13}{##1}}}
\expandafter\def\csname PYG@tok@s\endcsname{\def\PYG@tc##1{\textcolor[rgb]{0.25,0.44,0.63}{##1}}}
\expandafter\def\csname PYG@tok@kp\endcsname{\def\PYG@tc##1{\textcolor[rgb]{0.00,0.44,0.13}{##1}}}
\expandafter\def\csname PYG@tok@w\endcsname{\def\PYG@tc##1{\textcolor[rgb]{0.73,0.73,0.73}{##1}}}
\expandafter\def\csname PYG@tok@kt\endcsname{\def\PYG@tc##1{\textcolor[rgb]{0.56,0.13,0.00}{##1}}}
\expandafter\def\csname PYG@tok@sc\endcsname{\def\PYG@tc##1{\textcolor[rgb]{0.25,0.44,0.63}{##1}}}
\expandafter\def\csname PYG@tok@sb\endcsname{\def\PYG@tc##1{\textcolor[rgb]{0.25,0.44,0.63}{##1}}}
\expandafter\def\csname PYG@tok@k\endcsname{\let\PYG@bf=\textbf\def\PYG@tc##1{\textcolor[rgb]{0.00,0.44,0.13}{##1}}}
\expandafter\def\csname PYG@tok@se\endcsname{\let\PYG@bf=\textbf\def\PYG@tc##1{\textcolor[rgb]{0.25,0.44,0.63}{##1}}}
\expandafter\def\csname PYG@tok@sd\endcsname{\let\PYG@it=\textit\def\PYG@tc##1{\textcolor[rgb]{0.25,0.44,0.63}{##1}}}

\def\PYGZbs{\char`\\}
\def\PYGZus{\char`\_}
\def\PYGZob{\char`\{}
\def\PYGZcb{\char`\}}
\def\PYGZca{\char`\^}
\def\PYGZam{\char`\&}
\def\PYGZlt{\char`\<}
\def\PYGZgt{\char`\>}
\def\PYGZsh{\char`\#}
\def\PYGZpc{\char`\%}
\def\PYGZdl{\char`\$}
\def\PYGZhy{\char`\-}
\def\PYGZsq{\char`\'}
\def\PYGZdq{\char`\"}
\def\PYGZti{\char`\~}
% for compatibility with earlier versions
\def\PYGZat{@}
\def\PYGZlb{[}
\def\PYGZrb{]}
\makeatother

\begin{document}

\maketitle
\tableofcontents
\phantomsection\label{index::doc}


Contents:


\chapter{Contour Package}
\label{Contour:welcome-to-contour-s-documentation}\label{Contour:contour-package}\label{Contour::doc}

\section{\texttt{manage} Module}
\label{Contour:manage-module}\label{Contour:module-Contour.manage}\index{Contour.manage (module)}
Django's command-line utility for administrative tasks.


\section{\texttt{settings} Module}
\label{Contour:module-Contour.settings}\label{Contour:settings-module}\index{Contour.settings (module)}
Django settings for the Contour project.


\section{\texttt{urls} Module}
\label{Contour:module-Contour.urls}\label{Contour:urls-module}\index{Contour.urls (module)}
The urls module maps URL patterns of requested URLs to Django views.


\section{Subpackages}
\label{Contour:subpackages}

\subsection{contour Package}
\label{Contour.contour:contour-package}\label{Contour.contour::doc}

\subsubsection{\texttt{admin} Module}
\label{Contour.contour:admin-module}\label{Contour.contour:module-Contour.contour.admin}\index{Contour.contour.admin (module)}
Describes the representation of models in the admin interface.
\index{TrackAdmin (class in Contour.contour.admin)}

\begin{fulllineitems}
\phantomsection\label{Contour.contour:Contour.contour.admin.TrackAdmin}\pysiglinewithargsret{\strong{class }\code{Contour.contour.admin.}\bfcode{TrackAdmin}}{\emph{model}, \emph{admin\_site}}{}
Bases: \code{django.contrib.admin.options.ModelAdmin}

Describes the representation of tracks in the admin interface.
\index{inlines (Contour.contour.admin.TrackAdmin attribute)}

\begin{fulllineitems}
\phantomsection\label{Contour.contour:Contour.contour.admin.TrackAdmin.inlines}\pysigline{\bfcode{inlines}\strong{ = {[}\textless{}class `Contour.contour.admin.TrackImageInline'\textgreater{}{]}}}
\end{fulllineitems}

\index{media (Contour.contour.admin.TrackAdmin attribute)}

\begin{fulllineitems}
\phantomsection\label{Contour.contour:Contour.contour.admin.TrackAdmin.media}\pysigline{\bfcode{media}}
\end{fulllineitems}


\end{fulllineitems}

\index{TrackImageInline (class in Contour.contour.admin)}

\begin{fulllineitems}
\phantomsection\label{Contour.contour:Contour.contour.admin.TrackImageInline}\pysiglinewithargsret{\strong{class }\code{Contour.contour.admin.}\bfcode{TrackImageInline}}{\emph{parent\_model}, \emph{admin\_site}}{}
Bases: \code{django.contrib.admin.options.TabularInline}

Describes the representation of images in the track admin interface.
\index{media (Contour.contour.admin.TrackImageInline attribute)}

\begin{fulllineitems}
\phantomsection\label{Contour.contour:Contour.contour.admin.TrackImageInline.media}\pysigline{\bfcode{media}}
\end{fulllineitems}

\index{model (Contour.contour.admin.TrackImageInline attribute)}

\begin{fulllineitems}
\phantomsection\label{Contour.contour:Contour.contour.admin.TrackImageInline.model}\pysigline{\bfcode{model}}
alias of \code{TrackImage}

\end{fulllineitems}

\index{ordering (Contour.contour.admin.TrackImageInline attribute)}

\begin{fulllineitems}
\phantomsection\label{Contour.contour:Contour.contour.admin.TrackImageInline.ordering}\pysigline{\bfcode{ordering}\strong{ = {[}'order'{]}}}
\end{fulllineitems}


\end{fulllineitems}



\subsubsection{\texttt{forms} Module}
\label{Contour.contour:module-Contour.contour.forms}\label{Contour.contour:forms-module}\index{Contour.contour.forms (module)}
Forms which describe the asynchronous communication between server and client.
\index{DiscardSessionForm (class in Contour.contour.forms)}

\begin{fulllineitems}
\phantomsection\label{Contour.contour:Contour.contour.forms.DiscardSessionForm}\pysiglinewithargsret{\strong{class }\code{Contour.contour.forms.}\bfcode{DiscardSessionForm}}{\emph{data=None}, \emph{files=None}, \emph{auto\_id='id\_\%s'}, \emph{prefix=None}, \emph{initial=None}, \emph{error\_class=\textless{}class `django.forms.util.ErrorList'\textgreater{}}, \emph{label\_suffix=':'}, \emph{empty\_permitted=False}}{}
Bases: \code{django.forms.forms.Form}

Form used for the discarding of sessions.
\index{base\_fields (Contour.contour.forms.DiscardSessionForm attribute)}

\begin{fulllineitems}
\phantomsection\label{Contour.contour:Contour.contour.forms.DiscardSessionForm.base_fields}\pysigline{\bfcode{base\_fields}\strong{ = \{`discard\_session': \textless{}django.forms.fields.BooleanField object at 0x2ee0110\textgreater{}\}}}
\end{fulllineitems}

\index{media (Contour.contour.forms.DiscardSessionForm attribute)}

\begin{fulllineitems}
\phantomsection\label{Contour.contour:Contour.contour.forms.DiscardSessionForm.media}\pysigline{\bfcode{media}}
\end{fulllineitems}


\end{fulllineitems}

\index{FinishDrawingForm (class in Contour.contour.forms)}

\begin{fulllineitems}
\phantomsection\label{Contour.contour:Contour.contour.forms.FinishDrawingForm}\pysiglinewithargsret{\strong{class }\code{Contour.contour.forms.}\bfcode{FinishDrawingForm}}{\emph{data=None}, \emph{files=None}, \emph{auto\_id='id\_\%s'}, \emph{prefix=None}, \emph{initial=None}, \emph{error\_class=\textless{}class `django.forms.util.ErrorList'\textgreater{}}, \emph{label\_suffix=':'}, \emph{empty\_permitted=False}}{}
Bases: \code{django.forms.forms.Form}

Form used for the submission of drawings.
\index{base\_fields (Contour.contour.forms.FinishDrawingForm attribute)}

\begin{fulllineitems}
\phantomsection\label{Contour.contour:Contour.contour.forms.FinishDrawingForm.base_fields}\pysigline{\bfcode{base\_fields}\strong{ = \{`finish\_drawing': \textless{}django.forms.fields.BooleanField object at 0x2ee0090\textgreater{}\}}}
\end{fulllineitems}

\index{media (Contour.contour.forms.FinishDrawingForm attribute)}

\begin{fulllineitems}
\phantomsection\label{Contour.contour:Contour.contour.forms.FinishDrawingForm.media}\pysigline{\bfcode{media}}
\end{fulllineitems}


\end{fulllineitems}

\index{SaveSessionForm (class in Contour.contour.forms)}

\begin{fulllineitems}
\phantomsection\label{Contour.contour:Contour.contour.forms.SaveSessionForm}\pysiglinewithargsret{\strong{class }\code{Contour.contour.forms.}\bfcode{SaveSessionForm}}{\emph{data=None}, \emph{files=None}, \emph{auto\_id='id\_\%s'}, \emph{prefix=None}, \emph{initial=None}, \emph{error\_class=\textless{}class `django.forms.util.ErrorList'\textgreater{}}, \emph{label\_suffix=':'}, \emph{empty\_permitted=False}}{}
Bases: \code{django.forms.forms.Form}

Form used for the saving of track sessions (for highscore tracking).
\index{base\_fields (Contour.contour.forms.SaveSessionForm attribute)}

\begin{fulllineitems}
\phantomsection\label{Contour.contour:Contour.contour.forms.SaveSessionForm.base_fields}\pysigline{\bfcode{base\_fields}\strong{ = \{`name': \textless{}django.forms.fields.CharField object at 0x2ee0190\textgreater{}, `save\_session': \textless{}django.forms.fields.BooleanField object at 0x2ee0250\textgreater{}\}}}
\end{fulllineitems}

\index{media (Contour.contour.forms.SaveSessionForm attribute)}

\begin{fulllineitems}
\phantomsection\label{Contour.contour:Contour.contour.forms.SaveSessionForm.media}\pysigline{\bfcode{media}}
\end{fulllineitems}


\end{fulllineitems}

\index{SearchFlickrForm (class in Contour.contour.forms)}

\begin{fulllineitems}
\phantomsection\label{Contour.contour:Contour.contour.forms.SearchFlickrForm}\pysiglinewithargsret{\strong{class }\code{Contour.contour.forms.}\bfcode{SearchFlickrForm}}{\emph{data=None}, \emph{files=None}, \emph{auto\_id='id\_\%s'}, \emph{prefix=None}, \emph{initial=None}, \emph{error\_class=\textless{}class `django.forms.util.ErrorList'\textgreater{}}, \emph{label\_suffix=':'}, \emph{empty\_permitted=False}}{}
Bases: \code{django.forms.forms.Form}

Form used for the submitting the Flickr search query.
\index{base\_fields (Contour.contour.forms.SearchFlickrForm attribute)}

\begin{fulllineitems}
\phantomsection\label{Contour.contour:Contour.contour.forms.SearchFlickrForm.base_fields}\pysigline{\bfcode{base\_fields}\strong{ = \{`query': \textless{}django.forms.fields.CharField object at 0x2ee0450\textgreater{}, `search\_flickr': \textless{}django.forms.fields.BooleanField object at 0x2ee0590\textgreater{}, `sigma': \textless{}django.forms.fields.FloatField object at 0x2ee0610\textgreater{}\}}}
\end{fulllineitems}

\index{media (Contour.contour.forms.SearchFlickrForm attribute)}

\begin{fulllineitems}
\phantomsection\label{Contour.contour:Contour.contour.forms.SearchFlickrForm.media}\pysigline{\bfcode{media}}
\end{fulllineitems}


\end{fulllineitems}

\index{UploadFileForm (class in Contour.contour.forms)}

\begin{fulllineitems}
\phantomsection\label{Contour.contour:Contour.contour.forms.UploadFileForm}\pysiglinewithargsret{\strong{class }\code{Contour.contour.forms.}\bfcode{UploadFileForm}}{\emph{data=None}, \emph{files=None}, \emph{auto\_id='id\_\%s'}, \emph{prefix=None}, \emph{initial=None}, \emph{error\_class=\textless{}class `django.forms.util.ErrorList'\textgreater{}}, \emph{label\_suffix=':'}, \emph{empty\_permitted=False}}{}
Bases: \code{django.forms.forms.Form}

Form used for the upload of files.
\index{base\_fields (Contour.contour.forms.UploadFileForm attribute)}

\begin{fulllineitems}
\phantomsection\label{Contour.contour:Contour.contour.forms.UploadFileForm.base_fields}\pysigline{\bfcode{base\_fields}\strong{ = \{`file': \textless{}django.forms.fields.FileField object at 0x2ee02d0\textgreater{}, `sigma': \textless{}django.forms.fields.FloatField object at 0x2ee03d0\textgreater{}\}}}
\end{fulllineitems}

\index{media (Contour.contour.forms.UploadFileForm attribute)}

\begin{fulllineitems}
\phantomsection\label{Contour.contour:Contour.contour.forms.UploadFileForm.media}\pysigline{\bfcode{media}}
\end{fulllineitems}


\end{fulllineitems}



\subsubsection{\texttt{models} Module}
\label{Contour.contour:module-Contour.contour.models}\label{Contour.contour:models-module}\index{Contour.contour.models (module)}
Describes the models used in Contour.
\index{Drawing (class in Contour.contour.models)}

\begin{fulllineitems}
\phantomsection\label{Contour.contour:Contour.contour.models.Drawing}\pysiglinewithargsret{\strong{class }\code{Contour.contour.models.}\bfcode{Drawing}}{\emph{*args}, \emph{**kwargs}}{}
Bases: \code{django.db.models.base.Model}

Stores a drawing conducted by a player.
\index{Drawing.DoesNotExist}

\begin{fulllineitems}
\phantomsection\label{Contour.contour:Contour.contour.models.Drawing.DoesNotExist}\pysigline{\strong{exception }\bfcode{DoesNotExist}}
Bases: \code{django.core.exceptions.ObjectDoesNotExist}

\end{fulllineitems}

\index{Drawing.MultipleObjectsReturned}

\begin{fulllineitems}
\phantomsection\label{Contour.contour:Contour.contour.models.Drawing.MultipleObjectsReturned}\pysigline{\strong{exception }\code{Drawing.}\bfcode{MultipleObjectsReturned}}
Bases: \code{django.core.exceptions.MultipleObjectsReturned}

\end{fulllineitems}

\index{datetime (Contour.contour.models.Drawing attribute)}

\begin{fulllineitems}
\phantomsection\label{Contour.contour:Contour.contour.models.Drawing.datetime}\pysigline{\code{Drawing.}\bfcode{datetime}\strong{ = None}}
The modification time for the drawing.

\end{fulllineitems}

\index{delete() (Contour.contour.models.Drawing method)}

\begin{fulllineitems}
\phantomsection\label{Contour.contour:Contour.contour.models.Drawing.delete}\pysiglinewithargsret{\code{Drawing.}\bfcode{delete}}{\emph{*args}, \emph{**kwargs}}{}
Deletes the {\hyperref[Contour.contour:Contour.contour.models.Drawing]{\code{Drawing}}} object along with its associated files.

\end{fulllineitems}

\index{distance (Contour.contour.models.Drawing attribute)}

\begin{fulllineitems}
\phantomsection\label{Contour.contour:Contour.contour.models.Drawing.distance}\pysigline{\code{Drawing.}\bfcode{distance}\strong{ = None}}
The distance of between the drawing and the original (edge) image.

\end{fulllineitems}

\index{drawing (Contour.contour.models.Drawing attribute)}

\begin{fulllineitems}
\phantomsection\label{Contour.contour:Contour.contour.models.Drawing.drawing}\pysigline{\code{Drawing.}\bfcode{drawing}}
\code{django.db.models.ImageField} to the drawing on the filesystem.

\end{fulllineitems}

\index{get\_next\_by\_datetime() (Contour.contour.models.Drawing method)}

\begin{fulllineitems}
\phantomsection\label{Contour.contour:Contour.contour.models.Drawing.get_next_by_datetime}\pysiglinewithargsret{\code{Drawing.}\bfcode{get\_next\_by\_datetime}}{\emph{*moreargs}, \emph{**morekwargs}}{}
\end{fulllineitems}

\index{get\_previous\_by\_datetime() (Contour.contour.models.Drawing method)}

\begin{fulllineitems}
\phantomsection\label{Contour.contour:Contour.contour.models.Drawing.get_previous_by_datetime}\pysiglinewithargsret{\code{Drawing.}\bfcode{get\_previous\_by\_datetime}}{\emph{*moreargs}, \emph{**morekwargs}}{}
\end{fulllineitems}

\index{image (Contour.contour.models.Drawing attribute)}

\begin{fulllineitems}
\phantomsection\label{Contour.contour:Contour.contour.models.Drawing.image}\pysigline{\code{Drawing.}\bfcode{image}}
The reference to the {\hyperref[Contour.contour:Contour.contour.models.Image]{\code{Image}}} object which is the template for the drawing.

\end{fulllineitems}

\index{objects (Contour.contour.models.Drawing attribute)}

\begin{fulllineitems}
\phantomsection\label{Contour.contour:Contour.contour.models.Drawing.objects}\pysigline{\code{Drawing.}\bfcode{objects}\strong{ = \textless{}django.db.models.manager.Manager object at 0x3171e90\textgreater{}}}
\end{fulllineitems}

\index{player (Contour.contour.models.Drawing attribute)}

\begin{fulllineitems}
\phantomsection\label{Contour.contour:Contour.contour.models.Drawing.player}\pysigline{\code{Drawing.}\bfcode{player}}
The reference to the {\hyperref[Contour.contour:Contour.contour.models.Player]{\code{Player}}} object who was drawn the drawing.

\end{fulllineitems}

\index{score (Contour.contour.models.Drawing attribute)}

\begin{fulllineitems}
\phantomsection\label{Contour.contour:Contour.contour.models.Drawing.score}\pysigline{\code{Drawing.}\bfcode{score}\strong{ = None}}
The attained score for the drawing.

\end{fulllineitems}

\index{track\_session (Contour.contour.models.Drawing attribute)}

\begin{fulllineitems}
\phantomsection\label{Contour.contour:Contour.contour.models.Drawing.track_session}\pysigline{\code{Drawing.}\bfcode{track\_session}}
The reference to the {\hyperref[Contour.contour:Contour.contour.models.TrackSession]{\code{TrackSession}}} (optional).

\end{fulllineitems}

\index{track\_session\_index (Contour.contour.models.Drawing attribute)}

\begin{fulllineitems}
\phantomsection\label{Contour.contour:Contour.contour.models.Drawing.track_session_index}\pysigline{\code{Drawing.}\bfcode{track\_session\_index}\strong{ = None}}
The index of the drawing inside a {\hyperref[Contour.contour:Contour.contour.models.TrackSession]{\code{TrackSession}}}..

\end{fulllineitems}


\end{fulllineitems}

\index{Image (class in Contour.contour.models)}

\begin{fulllineitems}
\phantomsection\label{Contour.contour:Contour.contour.models.Image}\pysiglinewithargsret{\strong{class }\code{Contour.contour.models.}\bfcode{Image}}{\emph{*args}, \emph{**kwargs}}{}
Bases: \code{django.db.models.base.Model}

Stores a single image which serves as a template for {\hyperref[Contour.contour:Contour.contour.models.Drawing]{\code{Drawing}}} objects .
\index{Image.DoesNotExist}

\begin{fulllineitems}
\phantomsection\label{Contour.contour:Contour.contour.models.Image.DoesNotExist}\pysigline{\strong{exception }\bfcode{DoesNotExist}}
Bases: \code{django.core.exceptions.ObjectDoesNotExist}

\end{fulllineitems}

\index{Image.MultipleObjectsReturned}

\begin{fulllineitems}
\phantomsection\label{Contour.contour:Contour.contour.models.Image.MultipleObjectsReturned}\pysigline{\strong{exception }\code{Image.}\bfcode{MultipleObjectsReturned}}
Bases: \code{django.core.exceptions.MultipleObjectsReturned}

\end{fulllineitems}

\index{author (Contour.contour.models.Image attribute)}

\begin{fulllineitems}
\phantomsection\label{Contour.contour:Contour.contour.models.Image.author}\pysigline{\code{Image.}\bfcode{author}\strong{ = None}}
Optional author of the image (e.g. Pablo Picasso).

\end{fulllineitems}

\index{canny\_high\_threshold (Contour.contour.models.Image attribute)}

\begin{fulllineitems}
\phantomsection\label{Contour.contour:Contour.contour.models.Image.canny_high_threshold}\pysigline{\code{Image.}\bfcode{canny\_high\_threshold}\strong{ = None}}
The high threshold for the Canny edge detection.

\end{fulllineitems}

\index{canny\_low\_threshold (Contour.contour.models.Image attribute)}

\begin{fulllineitems}
\phantomsection\label{Contour.contour:Contour.contour.models.Image.canny_low_threshold}\pysigline{\code{Image.}\bfcode{canny\_low\_threshold}\strong{ = None}}
The low threshold for the Canny edge detection.

\end{fulllineitems}

\index{canny\_sigma (Contour.contour.models.Image attribute)}

\begin{fulllineitems}
\phantomsection\label{Contour.contour:Contour.contour.models.Image.canny_sigma}\pysigline{\code{Image.}\bfcode{canny\_sigma}\strong{ = None}}
The sigma value for the Gaussian blur which is performed before the Canny edge detection.

\end{fulllineitems}

\index{delete() (Contour.contour.models.Image method)}

\begin{fulllineitems}
\phantomsection\label{Contour.contour:Contour.contour.models.Image.delete}\pysiglinewithargsret{\code{Image.}\bfcode{delete}}{\emph{*args}, \emph{**kwargs}}{}
Deletes the {\hyperref[Contour.contour:Contour.contour.models.Image]{\code{Image}}} object along with its associated files.

\end{fulllineitems}

\index{dilated\_edge\_image (Contour.contour.models.Image attribute)}

\begin{fulllineitems}
\phantomsection\label{Contour.contour:Contour.contour.models.Image.dilated_edge_image}\pysigline{\code{Image.}\bfcode{dilated\_edge\_image}}
\code{django.db.models.ImageField} to the dilated edge image on the filesystem. The dilated edge image is used to assess the score of a drawing.

\end{fulllineitems}

\index{drawing\_set (Contour.contour.models.Image attribute)}

\begin{fulllineitems}
\phantomsection\label{Contour.contour:Contour.contour.models.Image.drawing_set}\pysigline{\code{Image.}\bfcode{drawing\_set}}
\end{fulllineitems}

\index{edge\_image (Contour.contour.models.Image attribute)}

\begin{fulllineitems}
\phantomsection\label{Contour.contour:Contour.contour.models.Image.edge_image}\pysigline{\code{Image.}\bfcode{edge\_image}}
\code{django.db.models.ImageField} to the edge image on the filesystem. The edge image is calculated automatically.

\end{fulllineitems}

\index{image (Contour.contour.models.Image attribute)}

\begin{fulllineitems}
\phantomsection\label{Contour.contour:Contour.contour.models.Image.image}\pysigline{\code{Image.}\bfcode{image}}
\code{django.db.models.ImageField} to the image on the filesystem.

\end{fulllineitems}

\index{max\_distance (Contour.contour.models.Image attribute)}

\begin{fulllineitems}
\phantomsection\label{Contour.contour:Contour.contour.models.Image.max_distance}\pysigline{\code{Image.}\bfcode{max\_distance}\strong{ = None}}
This is a value which is used for the score calculation.

\end{fulllineitems}

\index{objects (Contour.contour.models.Image attribute)}

\begin{fulllineitems}
\phantomsection\label{Contour.contour:Contour.contour.models.Image.objects}\pysigline{\code{Image.}\bfcode{objects}\strong{ = \textless{}django.db.models.manager.Manager object at 0x316e310\textgreater{}}}
\end{fulllineitems}

\index{title (Contour.contour.models.Image attribute)}

\begin{fulllineitems}
\phantomsection\label{Contour.contour:Contour.contour.models.Image.title}\pysigline{\code{Image.}\bfcode{title}\strong{ = None}}
The title of the image.

\end{fulllineitems}

\index{track\_set (Contour.contour.models.Image attribute)}

\begin{fulllineitems}
\phantomsection\label{Contour.contour:Contour.contour.models.Image.track_set}\pysigline{\code{Image.}\bfcode{track\_set}}
\end{fulllineitems}

\index{trackimage\_set (Contour.contour.models.Image attribute)}

\begin{fulllineitems}
\phantomsection\label{Contour.contour:Contour.contour.models.Image.trackimage_set}\pysigline{\code{Image.}\bfcode{trackimage\_set}}
\end{fulllineitems}

\index{url (Contour.contour.models.Image attribute)}

\begin{fulllineitems}
\phantomsection\label{Contour.contour:Contour.contour.models.Image.url}\pysigline{\code{Image.}\bfcode{url}\strong{ = None}}
Optional link to additional information.

\end{fulllineitems}


\end{fulllineitems}

\index{Player (class in Contour.contour.models)}

\begin{fulllineitems}
\phantomsection\label{Contour.contour:Contour.contour.models.Player}\pysiglinewithargsret{\strong{class }\code{Contour.contour.models.}\bfcode{Player}}{\emph{*args}, \emph{**kwargs}}{}
Bases: \code{django.db.models.base.Model}

Stores a player which can be associated to {\hyperref[Contour.contour:Contour.contour.models.TrackSession]{\code{TrackSession}}} and {\hyperref[Contour.contour:Contour.contour.models.Drawing]{\code{Drawing}}} objects.
\index{Player.DoesNotExist}

\begin{fulllineitems}
\phantomsection\label{Contour.contour:Contour.contour.models.Player.DoesNotExist}\pysigline{\strong{exception }\bfcode{DoesNotExist}}
Bases: \code{django.core.exceptions.ObjectDoesNotExist}

\end{fulllineitems}

\index{Player.MultipleObjectsReturned}

\begin{fulllineitems}
\phantomsection\label{Contour.contour:Contour.contour.models.Player.MultipleObjectsReturned}\pysigline{\strong{exception }\code{Player.}\bfcode{MultipleObjectsReturned}}
Bases: \code{django.core.exceptions.MultipleObjectsReturned}

\end{fulllineitems}

\index{drawing\_set (Contour.contour.models.Player attribute)}

\begin{fulllineitems}
\phantomsection\label{Contour.contour:Contour.contour.models.Player.drawing_set}\pysigline{\code{Player.}\bfcode{drawing\_set}}
\end{fulllineitems}

\index{name (Contour.contour.models.Player attribute)}

\begin{fulllineitems}
\phantomsection\label{Contour.contour:Contour.contour.models.Player.name}\pysigline{\code{Player.}\bfcode{name}\strong{ = None}}
The name of the player.

\end{fulllineitems}

\index{objects (Contour.contour.models.Player attribute)}

\begin{fulllineitems}
\phantomsection\label{Contour.contour:Contour.contour.models.Player.objects}\pysigline{\code{Player.}\bfcode{objects}\strong{ = \textless{}django.db.models.manager.Manager object at 0x316efd0\textgreater{}}}
\end{fulllineitems}

\index{tracksession\_set (Contour.contour.models.Player attribute)}

\begin{fulllineitems}
\phantomsection\label{Contour.contour:Contour.contour.models.Player.tracksession_set}\pysigline{\code{Player.}\bfcode{tracksession\_set}}
\end{fulllineitems}


\end{fulllineitems}

\index{Track (class in Contour.contour.models)}

\begin{fulllineitems}
\phantomsection\label{Contour.contour:Contour.contour.models.Track}\pysiglinewithargsret{\strong{class }\code{Contour.contour.models.}\bfcode{Track}}{\emph{*args}, \emph{**kwargs}}{}
Bases: \code{django.db.models.base.Model}

Stores a track which is a set of {\hyperref[Contour.contour:Contour.contour.models.Image]{\code{Image}}} objects  supposed to be drawn in succession.
\index{Track.DoesNotExist}

\begin{fulllineitems}
\phantomsection\label{Contour.contour:Contour.contour.models.Track.DoesNotExist}\pysigline{\strong{exception }\bfcode{DoesNotExist}}
Bases: \code{django.core.exceptions.ObjectDoesNotExist}

\end{fulllineitems}

\index{Track.MultipleObjectsReturned}

\begin{fulllineitems}
\phantomsection\label{Contour.contour:Contour.contour.models.Track.MultipleObjectsReturned}\pysigline{\strong{exception }\code{Track.}\bfcode{MultipleObjectsReturned}}
Bases: \code{django.core.exceptions.MultipleObjectsReturned}

\end{fulllineitems}

\index{images (Contour.contour.models.Track attribute)}

\begin{fulllineitems}
\phantomsection\label{Contour.contour:Contour.contour.models.Track.images}\pysigline{\code{Track.}\bfcode{images}}
\end{fulllineitems}

\index{objects (Contour.contour.models.Track attribute)}

\begin{fulllineitems}
\phantomsection\label{Contour.contour:Contour.contour.models.Track.objects}\pysigline{\code{Track.}\bfcode{objects}\strong{ = \textless{}django.db.models.manager.Manager object at 0x316e890\textgreater{}}}
\end{fulllineitems}

\index{trackimage\_set (Contour.contour.models.Track attribute)}

\begin{fulllineitems}
\phantomsection\label{Contour.contour:Contour.contour.models.Track.trackimage_set}\pysigline{\code{Track.}\bfcode{trackimage\_set}}
\end{fulllineitems}

\index{tracksession\_set (Contour.contour.models.Track attribute)}

\begin{fulllineitems}
\phantomsection\label{Contour.contour:Contour.contour.models.Track.tracksession_set}\pysigline{\code{Track.}\bfcode{tracksession\_set}}
\end{fulllineitems}


\end{fulllineitems}

\index{TrackImage (class in Contour.contour.models)}

\begin{fulllineitems}
\phantomsection\label{Contour.contour:Contour.contour.models.TrackImage}\pysiglinewithargsret{\strong{class }\code{Contour.contour.models.}\bfcode{TrackImage}}{\emph{*args}, \emph{**kwargs}}{}
Bases: \code{django.db.models.base.Model}

Stores a many-to-many (n-to-n) relation between {\hyperref[Contour.contour:Contour.contour.models.Track]{\code{Track}}} objects  and {\hyperref[Contour.contour:Contour.contour.models.Image]{\code{Image}}} objects.
\index{TrackImage.DoesNotExist}

\begin{fulllineitems}
\phantomsection\label{Contour.contour:Contour.contour.models.TrackImage.DoesNotExist}\pysigline{\strong{exception }\bfcode{DoesNotExist}}
Bases: \code{django.core.exceptions.ObjectDoesNotExist}

\end{fulllineitems}

\index{TrackImage.MultipleObjectsReturned}

\begin{fulllineitems}
\phantomsection\label{Contour.contour:Contour.contour.models.TrackImage.MultipleObjectsReturned}\pysigline{\strong{exception }\code{TrackImage.}\bfcode{MultipleObjectsReturned}}
Bases: \code{django.core.exceptions.MultipleObjectsReturned}

\end{fulllineitems}

\index{image (Contour.contour.models.TrackImage attribute)}

\begin{fulllineitems}
\phantomsection\label{Contour.contour:Contour.contour.models.TrackImage.image}\pysigline{\code{TrackImage.}\bfcode{image}}
The reference to the {\hyperref[Contour.contour:Contour.contour.models.Image]{\code{Image}}} object.

\end{fulllineitems}

\index{objects (Contour.contour.models.TrackImage attribute)}

\begin{fulllineitems}
\phantomsection\label{Contour.contour:Contour.contour.models.TrackImage.objects}\pysigline{\code{TrackImage.}\bfcode{objects}\strong{ = \textless{}django.db.models.manager.Manager object at 0x316ed50\textgreater{}}}
\end{fulllineitems}

\index{order (Contour.contour.models.TrackImage attribute)}

\begin{fulllineitems}
\phantomsection\label{Contour.contour:Contour.contour.models.TrackImage.order}\pysigline{\code{TrackImage.}\bfcode{order}\strong{ = None}}
The index of a {\hyperref[Contour.contour:Contour.contour.models.Image]{\code{Image}}} object inside a {\hyperref[Contour.contour:Contour.contour.models.Track]{\code{Track}}}.

\end{fulllineitems}

\index{ordering (Contour.contour.models.TrackImage attribute)}

\begin{fulllineitems}
\phantomsection\label{Contour.contour:Contour.contour.models.TrackImage.ordering}\pysigline{\code{TrackImage.}\bfcode{ordering}\strong{ = {[}'order'{]}}}
\end{fulllineitems}

\index{track (Contour.contour.models.TrackImage attribute)}

\begin{fulllineitems}
\phantomsection\label{Contour.contour:Contour.contour.models.TrackImage.track}\pysigline{\code{TrackImage.}\bfcode{track}}
The reference to the {\hyperref[Contour.contour:Contour.contour.models.Track]{\code{Track}}} object.

\end{fulllineitems}


\end{fulllineitems}

\index{TrackSession (class in Contour.contour.models)}

\begin{fulllineitems}
\phantomsection\label{Contour.contour:Contour.contour.models.TrackSession}\pysiglinewithargsret{\strong{class }\code{Contour.contour.models.}\bfcode{TrackSession}}{\emph{*args}, \emph{**kwargs}}{}
Bases: \code{django.db.models.base.Model}

Stores a track sesssion which stores the specific track session of a player.
\index{TrackSession.DoesNotExist}

\begin{fulllineitems}
\phantomsection\label{Contour.contour:Contour.contour.models.TrackSession.DoesNotExist}\pysigline{\strong{exception }\bfcode{DoesNotExist}}
Bases: \code{django.core.exceptions.ObjectDoesNotExist}

\end{fulllineitems}

\index{TrackSession.MultipleObjectsReturned}

\begin{fulllineitems}
\phantomsection\label{Contour.contour:Contour.contour.models.TrackSession.MultipleObjectsReturned}\pysigline{\strong{exception }\code{TrackSession.}\bfcode{MultipleObjectsReturned}}
Bases: \code{django.core.exceptions.MultipleObjectsReturned}

\end{fulllineitems}

\index{datetime (Contour.contour.models.TrackSession attribute)}

\begin{fulllineitems}
\phantomsection\label{Contour.contour:Contour.contour.models.TrackSession.datetime}\pysigline{\code{TrackSession.}\bfcode{datetime}\strong{ = None}}
The creation time of the track session.

\end{fulllineitems}

\index{drawing\_set (Contour.contour.models.TrackSession attribute)}

\begin{fulllineitems}
\phantomsection\label{Contour.contour:Contour.contour.models.TrackSession.drawing_set}\pysigline{\code{TrackSession.}\bfcode{drawing\_set}}
\end{fulllineitems}

\index{get\_next\_by\_datetime() (Contour.contour.models.TrackSession method)}

\begin{fulllineitems}
\phantomsection\label{Contour.contour:Contour.contour.models.TrackSession.get_next_by_datetime}\pysiglinewithargsret{\code{TrackSession.}\bfcode{get\_next\_by\_datetime}}{\emph{*moreargs}, \emph{**morekwargs}}{}
\end{fulllineitems}

\index{get\_previous\_by\_datetime() (Contour.contour.models.TrackSession method)}

\begin{fulllineitems}
\phantomsection\label{Contour.contour:Contour.contour.models.TrackSession.get_previous_by_datetime}\pysiglinewithargsret{\code{TrackSession.}\bfcode{get\_previous\_by\_datetime}}{\emph{*moreargs}, \emph{**morekwargs}}{}
\end{fulllineitems}

\index{objects (Contour.contour.models.TrackSession attribute)}

\begin{fulllineitems}
\phantomsection\label{Contour.contour:Contour.contour.models.TrackSession.objects}\pysigline{\code{TrackSession.}\bfcode{objects}\strong{ = \textless{}django.db.models.manager.Manager object at 0x31716d0\textgreater{}}}
\end{fulllineitems}

\index{player (Contour.contour.models.TrackSession attribute)}

\begin{fulllineitems}
\phantomsection\label{Contour.contour:Contour.contour.models.TrackSession.player}\pysigline{\code{TrackSession.}\bfcode{player}}
The reference to a {\hyperref[Contour.contour:Contour.contour.models.Player]{\code{Player}}} object.

\end{fulllineitems}

\index{score (Contour.contour.models.TrackSession attribute)}

\begin{fulllineitems}
\phantomsection\label{Contour.contour:Contour.contour.models.TrackSession.score}\pysigline{\code{TrackSession.}\bfcode{score}\strong{ = None}}
The attained score.

\end{fulllineitems}

\index{session\_key (Contour.contour.models.TrackSession attribute)}

\begin{fulllineitems}
\phantomsection\label{Contour.contour:Contour.contour.models.TrackSession.session_key}\pysigline{\code{TrackSession.}\bfcode{session\_key}\strong{ = None}}
The associated session key of the user associated.

\end{fulllineitems}

\index{track (Contour.contour.models.TrackSession attribute)}

\begin{fulllineitems}
\phantomsection\label{Contour.contour:Contour.contour.models.TrackSession.track}\pysigline{\code{TrackSession.}\bfcode{track}}
The reference to a {\hyperref[Contour.contour:Contour.contour.models.Track]{\code{Track}}} object.

\end{fulllineitems}


\end{fulllineitems}



\subsubsection{\texttt{set\_metrics} Module}
\label{Contour.contour:module-Contour.contour.set_metrics}\label{Contour.contour:set-metrics-module}\index{Contour.contour.set\_metrics (module)}\index{binary\_find\_boundaries() (in module Contour.contour.set\_metrics)}

\begin{fulllineitems}
\phantomsection\label{Contour.contour:Contour.contour.set_metrics.binary_find_boundaries}\pysiglinewithargsret{\code{Contour.contour.set\_metrics.}\bfcode{binary\_find\_boundaries}}{\emph{image}}{}
\end{fulllineitems}

\index{hausdorff\_distance() (in module Contour.contour.set\_metrics)}

\begin{fulllineitems}
\phantomsection\label{Contour.contour:Contour.contour.set_metrics.hausdorff_distance}\pysiglinewithargsret{\code{Contour.contour.set\_metrics.}\bfcode{hausdorff\_distance}}{\emph{a}, \emph{b}}{}
Calculate the Hausdorff distance {[}1{]} between two sets.
\begin{quote}\begin{description}
\item[{Parameters}] \leavevmode\begin{itemize}
\item {} 
\textbf{a} (\emph{ndarray.}) -- Array containing the coordinates of \code{N} points in an \code{M} dimensional space.

\item {} 
\textbf{b} (\emph{ndarray.}) -- Array containing the coordinates of \code{N} points in an \code{M} dimensional space.

\end{itemize}

\item[{Returns}] \leavevmode
float -- The Hausdorff distance between the sets represented by \code{a} and \code{b} using Euclidian distance to calculate the distance between members of the sets.

\end{description}\end{quote}

\end{fulllineitems}

\index{hausdorff\_distance\_region() (in module Contour.contour.set\_metrics)}

\begin{fulllineitems}
\phantomsection\label{Contour.contour:Contour.contour.set_metrics.hausdorff_distance_region}\pysiglinewithargsret{\code{Contour.contour.set\_metrics.}\bfcode{hausdorff\_distance\_region}}{\emph{a}, \emph{b}}{}
Calculate the Hausdorff distance {[}2{]} between two binary images.
\begin{quote}\begin{description}
\item[{Parameters}] \leavevmode\begin{itemize}
\item {} 
\textbf{a} (\emph{ndarray, dtype=bool.}) -- Array where \code{True} represents a point that is included in a set of points. Both arrays must have the same shape.

\item {} 
\textbf{b} (\emph{ndarray, dtype=bool.}) -- Array where \code{True} represents a point that is included in a set of points. Both arrays must have the same shape.

\end{itemize}

\item[{Returns}] \leavevmode
float -- The Hausdorff distance between the sets represented by \code{a} and \code{b} using Euclidian distance to calculate the distance between members of the sets.

\end{description}\end{quote}

\end{fulllineitems}



\subsubsection{\texttt{urls} Module}
\label{Contour.contour:module-Contour.contour.urls}\label{Contour.contour:urls-module}\index{Contour.contour.urls (module)}
The urls module maps URL patterns of requested URLs to Django views.


\subsubsection{\texttt{util} Module}
\label{Contour.contour:module-Contour.contour.util}\label{Contour.contour:util-module}\index{Contour.contour.util (module)}
A module containing helper functions.
\index{slugify() (in module Contour.contour.util)}

\begin{fulllineitems}
\phantomsection\label{Contour.contour:Contour.contour.util.slugify}\pysiglinewithargsret{\code{Contour.contour.util.}\bfcode{slugify}}{\emph{value}}{}
Normalizes string, converts to lowercase, removes non-alpha characters,
and converts spaces to hyphens.

From Django's ``django/template/defaultfilters.py''.
\begin{quote}\begin{description}
\item[{Parameters}] \leavevmode
\textbf{value} (\emph{string.}) -- The unsanitized string.

\item[{Returns}] \leavevmode
string -- The sanitized string.

\end{description}\end{quote}

\end{fulllineitems}



\subsubsection{\texttt{views} Module}
\label{Contour.contour:module-Contour.contour.views}\label{Contour.contour:views-module}\index{Contour.contour.views (module)}
Describes the views used in Contour.
\index{canvas() (in module Contour.contour.views)}

\begin{fulllineitems}
\phantomsection\label{Contour.contour:Contour.contour.views.canvas}\pysiglinewithargsret{\code{Contour.contour.views.}\bfcode{canvas}}{\emph{request}, \emph{id=None}}{}
This is the view function for a single drawing canvas. It is called for the file upload and Flickr game modes.
\begin{quote}\begin{description}
\item[{Parameters}] \leavevmode\begin{itemize}
\item {} 
\textbf{request} (\code{django.http.HttpRequest}.) -- The request object containing the user request.

\item {} 
\textbf{id} (\emph{int.}) -- The id of the requested {\hyperref[Contour.contour:Contour.contour.models.Image]{\code{models.Image}}}.

\end{itemize}

\item[{Returns}] \leavevmode
\code{django.http.HttpResponse} -- The rendered template as response.

\end{description}\end{quote}

\end{fulllineitems}

\index{check\_session() (in module Contour.contour.views)}

\begin{fulllineitems}
\phantomsection\label{Contour.contour:Contour.contour.views.check_session}\pysiglinewithargsret{\code{Contour.contour.views.}\bfcode{check\_session}}{\emph{request}, \emph{view\_name=None}, \emph{id=None}}{}
Checks if the requested URL is in canon with the currently running session. The user will be asked if he wants to discad his session if there's a discrepancy.
\begin{quote}\begin{description}
\item[{Parameters}] \leavevmode\begin{itemize}
\item {} 
\textbf{request} (\code{django.http.HttpRequest}.) -- The request object containing the user request.

\item {} 
\textbf{view\_name} (\emph{string.}) -- The name of the requested view to which the session should be associated.

\item {} 
\textbf{id} (\emph{int.}) -- The id of the requested {\hyperref[Contour.contour:Contour.contour.models.Image]{\code{models.Image}}} or {\hyperref[Contour.contour:Contour.contour.models.Track]{\code{models.Track}}}.

\end{itemize}

\item[{Returns}] \leavevmode
\code{django.http.HttpResponse} -- The rendered template as response.

\end{description}\end{quote}

\end{fulllineitems}

\index{clear\_session() (in module Contour.contour.views)}

\begin{fulllineitems}
\phantomsection\label{Contour.contour:Contour.contour.views.clear_session}\pysiglinewithargsret{\code{Contour.contour.views.}\bfcode{clear\_session}}{\emph{request}}{}
Clears all varibles of the user session.
\begin{quote}\begin{description}
\item[{Parameters}] \leavevmode
\textbf{request} (\code{django.http.HttpRequest}.) -- The request object containing the user session.

\end{description}\end{quote}

\end{fulllineitems}

\index{create\_session() (in module Contour.contour.views)}

\begin{fulllineitems}
\phantomsection\label{Contour.contour:Contour.contour.views.create_session}\pysiglinewithargsret{\code{Contour.contour.views.}\bfcode{create\_session}}{\emph{request}, \emph{view\_name}, \emph{id}}{}
Creates a user session.
\begin{quote}\begin{description}
\item[{Parameters}] \leavevmode\begin{itemize}
\item {} 
\textbf{request} (\code{django.http.HttpRequest}.) -- The request object containing the user request.

\item {} 
\textbf{view\_name} (\emph{string.}) -- The name of the view to which the session should be associated.

\item {} 
\textbf{id} (\emph{int.}) -- The id of the {\hyperref[Contour.contour:Contour.contour.models.Image]{\code{models.Image}}} or {\hyperref[Contour.contour:Contour.contour.models.Track]{\code{models.Track}}}.

\end{itemize}

\end{description}\end{quote}

\end{fulllineitems}

\index{destroy\_session() (in module Contour.contour.views)}

\begin{fulllineitems}
\phantomsection\label{Contour.contour:Contour.contour.views.destroy_session}\pysiglinewithargsret{\code{Contour.contour.views.}\bfcode{destroy\_session}}{\emph{request}}{}
Destroys a currently running user session if such a request has been sent.
\begin{quote}\begin{description}
\item[{Parameters}] \leavevmode
\textbf{request} (\code{django.http.HttpRequest}.) -- The request object containing the user request.

\item[{Returns}] \leavevmode
bool -- \emph{True} if the session was cleared, otherwise \emph{None}.

\end{description}\end{quote}

\end{fulllineitems}

\index{drawing() (in module Contour.contour.views)}

\begin{fulllineitems}
\phantomsection\label{Contour.contour:Contour.contour.views.drawing}\pysiglinewithargsret{\code{Contour.contour.views.}\bfcode{drawing}}{\emph{request}, \emph{id}}{}
This is the view function to view the score summary of single drawings.
\begin{quote}\begin{description}
\item[{Parameters}] \leavevmode\begin{itemize}
\item {} 
\textbf{request} (\code{django.http.HttpRequest}.) -- The request object containing the user request.

\item {} 
\textbf{id} (\emph{int.}) -- The id of the requested {\hyperref[Contour.contour:Contour.contour.models.Drawing]{\code{models.Drawing}}}.

\end{itemize}

\item[{Returns}] \leavevmode
\code{django.http.HttpResponse} -- The rendered template as response.

\end{description}\end{quote}

\end{fulllineitems}

\index{get\_player() (in module Contour.contour.views)}

\begin{fulllineitems}
\phantomsection\label{Contour.contour:Contour.contour.views.get_player}\pysiglinewithargsret{\code{Contour.contour.views.}\bfcode{get\_player}}{\emph{name}}{}
Returns a {\hyperref[Contour.contour:Contour.contour.models.Player]{\code{models.Player}}} object. A new player will be created if the requested player doesn't exist.
\begin{quote}\begin{description}
\item[{Parameters}] \leavevmode
\textbf{view\_name} (\emph{string.}) -- The name of the requested player.

\item[{Returns}] \leavevmode
\code{models.Player} -- The requested player.

\end{description}\end{quote}

\end{fulllineitems}

\index{handle\_finished\_drawing() (in module Contour.contour.views)}

\begin{fulllineitems}
\phantomsection\label{Contour.contour:Contour.contour.views.handle_finished_drawing}\pysiglinewithargsret{\code{Contour.contour.views.}\bfcode{handle\_finished\_drawing}}{\emph{request}}{}
This function is called as soon as the user finishes his drawing. It saves and associates his drawing to the running track session. It also assesses the score of the drawing.
\begin{quote}\begin{description}
\item[{Parameters}] \leavevmode
\textbf{request} (\code{django.http.HttpRequest}.) -- The request object containing the user request.

\item[{Returns}] \leavevmode
\code{models.Drawing} -- The created drawing object.

\end{description}\end{quote}

\end{fulllineitems}

\index{handle\_flickr\_search() (in module Contour.contour.views)}

\begin{fulllineitems}
\phantomsection\label{Contour.contour:Contour.contour.views.handle_flickr_search}\pysiglinewithargsret{\code{Contour.contour.views.}\bfcode{handle\_flickr\_search}}{\emph{request}, \emph{form}}{}
This function is called as soon as the user submits a Flickr search query. It saves the found image on the filesystem.
\begin{quote}\begin{description}
\item[{Parameters}] \leavevmode
\textbf{request} (\code{django.http.HttpRequest}.) -- The request object containing the user request.

\item[{Returns}] \leavevmode
\code{models.Image} -- The created image object.

\end{description}\end{quote}

\end{fulllineitems}

\index{handle\_uploaded\_file() (in module Contour.contour.views)}

\begin{fulllineitems}
\phantomsection\label{Contour.contour:Contour.contour.views.handle_uploaded_file}\pysiglinewithargsret{\code{Contour.contour.views.}\bfcode{handle\_uploaded\_file}}{\emph{request}, \emph{form}}{}
This function is called as soon as the user uploads a file. It saves his image on the filesystem.
\begin{quote}\begin{description}
\item[{Parameters}] \leavevmode
\textbf{request} (\code{django.http.HttpRequest}.) -- The request object containing the user request.

\item[{Returns}] \leavevmode
\code{models.Image} -- The created image object.

\end{description}\end{quote}

\end{fulllineitems}

\index{index() (in module Contour.contour.views)}

\begin{fulllineitems}
\phantomsection\label{Contour.contour:Contour.contour.views.index}\pysiglinewithargsret{\code{Contour.contour.views.}\bfcode{index}}{\emph{request}}{}
This is the view function for the home page.
\begin{quote}\begin{description}
\item[{Parameters}] \leavevmode
\textbf{request} (\code{django.http.HttpRequest}.) -- The request object containing the user request.

\item[{Returns}] \leavevmode
\code{django.http.HttpResponse} -- The rendered template as response.

\end{description}\end{quote}

\end{fulllineitems}

\index{process\_image() (in module Contour.contour.views)}

\begin{fulllineitems}
\phantomsection\label{Contour.contour:Contour.contour.views.process_image}\pysiglinewithargsret{\code{Contour.contour.views.}\bfcode{process\_image}}{\emph{request}, \emph{image}}{}
Creates an edge image and calculates the values needed for the score calculation if necessary. This function is called as soon as an image is requested.
\begin{quote}\begin{description}
\item[{Parameters}] \leavevmode\begin{itemize}
\item {} 
\textbf{request} (\code{django.http.HttpRequest}.) -- The request object containing the user request.

\item {} 
\textbf{image} (\code{models.Image}.) -- The image to be processed.

\end{itemize}

\end{description}\end{quote}

\end{fulllineitems}

\index{save\_session() (in module Contour.contour.views)}

\begin{fulllineitems}
\phantomsection\label{Contour.contour:Contour.contour.views.save_session}\pysiglinewithargsret{\code{Contour.contour.views.}\bfcode{save\_session}}{\emph{request}}{}
Saves a track session. This function is called as soon as the player chooses to save his scores.
\begin{quote}\begin{description}
\item[{Parameters}] \leavevmode
\textbf{request} (\code{django.http.HttpRequest}.) -- The request object containing the user request.

\end{description}\end{quote}

\end{fulllineitems}

\index{session() (in module Contour.contour.views)}

\begin{fulllineitems}
\phantomsection\label{Contour.contour:Contour.contour.views.session}\pysiglinewithargsret{\code{Contour.contour.views.}\bfcode{session}}{\emph{request}, \emph{id}}{}
This is the view function to view the score summary of track sessions.
\begin{quote}\begin{description}
\item[{Parameters}] \leavevmode\begin{itemize}
\item {} 
\textbf{request} (\code{django.http.HttpRequest}.) -- The request object containing the user request.

\item {} 
\textbf{id} (\emph{int.}) -- The id of the requested {\hyperref[Contour.contour:Contour.contour.models.TrackSession]{\code{models.TrackSession}}}.

\end{itemize}

\item[{Returns}] \leavevmode
\code{django.http.HttpResponse} -- The rendered template as response.

\end{description}\end{quote}

\end{fulllineitems}

\index{track() (in module Contour.contour.views)}

\begin{fulllineitems}
\phantomsection\label{Contour.contour:Contour.contour.views.track}\pysiglinewithargsret{\code{Contour.contour.views.}\bfcode{track}}{\emph{request}, \emph{id}}{}
This is the view function for track sessions.
\begin{quote}\begin{description}
\item[{Parameters}] \leavevmode\begin{itemize}
\item {} 
\textbf{request} (\code{django.http.HttpRequest}.) -- The request object containing the user request.

\item {} 
\textbf{id} (\emph{int.}) -- The id of the requested {\hyperref[Contour.contour:Contour.contour.models.Track]{\code{models.Track}}}.

\end{itemize}

\item[{Returns}] \leavevmode
\code{django.http.HttpResponse} -- The rendered template as response.

\end{description}\end{quote}

\end{fulllineitems}



\subsubsection{Subpackages}
\label{Contour.contour:subpackages}

\paragraph{management Package}
\label{Contour.contour.management:management-package}\label{Contour.contour.management::doc}

\subparagraph{Subpackages}
\label{Contour.contour.management:subpackages}

\subparagraph{commands Package}
\label{Contour.contour.management.commands::doc}\label{Contour.contour.management.commands:commands-package}

\subparagraph{\texttt{cleanup} Module}
\label{Contour.contour.management.commands:module-Contour.contour.management.commands.cleanup}\label{Contour.contour.management.commands:cleanup-module}\index{Contour.contour.management.commands.cleanup (module)}
Cleans the database and filesystem from unreferenced items. This command should be run on a regular schedule.

The command is invoked by the following command: \code{python manage.py cleanup}
\index{Command (class in Contour.contour.management.commands.cleanup)}

\begin{fulllineitems}
\phantomsection\label{Contour.contour.management.commands:Contour.contour.management.commands.cleanup.Command}\pysigline{\strong{class }\code{Contour.contour.management.commands.cleanup.}\bfcode{Command}}
Bases: \code{django.core.management.base.BaseCommand}
\index{handle() (Contour.contour.management.commands.cleanup.Command method)}

\begin{fulllineitems}
\phantomsection\label{Contour.contour.management.commands:Contour.contour.management.commands.cleanup.Command.handle}\pysiglinewithargsret{\bfcode{handle}}{\emph{*args}, \emph{**options}}{}
\end{fulllineitems}

\index{help (Contour.contour.management.commands.cleanup.Command attribute)}

\begin{fulllineitems}
\phantomsection\label{Contour.contour.management.commands:Contour.contour.management.commands.cleanup.Command.help}\pysigline{\bfcode{help}\strong{ = `Cleans the database and filesystem from unreferenced items.'}}
\end{fulllineitems}


\end{fulllineitems}



\subparagraph{\texttt{delete\_edge\_images} Module}
\label{Contour.contour.management.commands:delete-edge-images-module}\label{Contour.contour.management.commands:module-Contour.contour.management.commands.delete_edge_images}\index{Contour.contour.management.commands.delete\_edge\_images (module)}
Deletes all edge images and precomputed values. The computations will be regenerated when needed.

The command is invoked by the following command: \code{python manage.py delete\_edge\_images}
\index{Command (class in Contour.contour.management.commands.delete\_edge\_images)}

\begin{fulllineitems}
\phantomsection\label{Contour.contour.management.commands:Contour.contour.management.commands.delete_edge_images.Command}\pysigline{\strong{class }\code{Contour.contour.management.commands.delete\_edge\_images.}\bfcode{Command}}
Bases: \code{django.core.management.base.BaseCommand}
\index{handle() (Contour.contour.management.commands.delete\_edge\_images.Command method)}

\begin{fulllineitems}
\phantomsection\label{Contour.contour.management.commands:Contour.contour.management.commands.delete_edge_images.Command.handle}\pysiglinewithargsret{\bfcode{handle}}{\emph{*args}, \emph{**options}}{}
\end{fulllineitems}

\index{help (Contour.contour.management.commands.delete\_edge\_images.Command attribute)}

\begin{fulllineitems}
\phantomsection\label{Contour.contour.management.commands:Contour.contour.management.commands.delete_edge_images.Command.help}\pysigline{\bfcode{help}\strong{ = `Deletes all edge images.'}}
\end{fulllineitems}


\end{fulllineitems}



\subparagraph{\texttt{delete\_user\_content} Module}
\label{Contour.contour.management.commands:delete-user-content-module}\label{Contour.contour.management.commands:module-Contour.contour.management.commands.delete_user_content}\index{Contour.contour.management.commands.delete\_user\_content (module)}
Deletes all user content and therefore resets the installation. Handle with care.

The command is invoked by the following command: \code{python manage.py delete\_user\_content}
\index{Command (class in Contour.contour.management.commands.delete\_user\_content)}

\begin{fulllineitems}
\phantomsection\label{Contour.contour.management.commands:Contour.contour.management.commands.delete_user_content.Command}\pysigline{\strong{class }\code{Contour.contour.management.commands.delete\_user\_content.}\bfcode{Command}}
Bases: \code{django.core.management.base.BaseCommand}
\index{handle() (Contour.contour.management.commands.delete\_user\_content.Command method)}

\begin{fulllineitems}
\phantomsection\label{Contour.contour.management.commands:Contour.contour.management.commands.delete_user_content.Command.handle}\pysiglinewithargsret{\bfcode{handle}}{\emph{*args}, \emph{**options}}{}
\end{fulllineitems}

\index{help (Contour.contour.management.commands.delete\_user\_content.Command attribute)}

\begin{fulllineitems}
\phantomsection\label{Contour.contour.management.commands:Contour.contour.management.commands.delete_user_content.Command.help}\pysigline{\bfcode{help}\strong{ = `Deletes all user content. Handle with care.'}}
\end{fulllineitems}


\end{fulllineitems}



\chapter{Indices and tables}
\label{index:indices-and-tables}\begin{itemize}
\item {} 
\emph{genindex}

\item {} 
\emph{modindex}

\item {} 
\emph{search}

\end{itemize}


\renewcommand{\indexname}{Python Module Index}
\begin{theindex}
\def\bigletter#1{{\Large\sffamily#1}\nopagebreak\vspace{1mm}}
\bigletter{c}
\item {\texttt{Contour.contour.admin}}, \pageref{Contour.contour:module-Contour.contour.admin}
\item {\texttt{Contour.contour.forms}}, \pageref{Contour.contour:module-Contour.contour.forms}
\item {\texttt{Contour.contour.management.commands.cleanup}}, \pageref{Contour.contour.management.commands:module-Contour.contour.management.commands.cleanup}
\item {\texttt{Contour.contour.management.commands.delete\_edge\_images}}, \pageref{Contour.contour.management.commands:module-Contour.contour.management.commands.delete_edge_images}
\item {\texttt{Contour.contour.management.commands.delete\_user\_content}}, \pageref{Contour.contour.management.commands:module-Contour.contour.management.commands.delete_user_content}
\item {\texttt{Contour.contour.models}}, \pageref{Contour.contour:module-Contour.contour.models}
\item {\texttt{Contour.contour.set\_metrics}}, \pageref{Contour.contour:module-Contour.contour.set_metrics}
\item {\texttt{Contour.contour.urls}}, \pageref{Contour.contour:module-Contour.contour.urls}
\item {\texttt{Contour.contour.util}}, \pageref{Contour.contour:module-Contour.contour.util}
\item {\texttt{Contour.contour.views}}, \pageref{Contour.contour:module-Contour.contour.views}
\item {\texttt{Contour.manage}}, \pageref{Contour:module-Contour.manage}
\item {\texttt{Contour.settings}}, \pageref{Contour:module-Contour.settings}
\item {\texttt{Contour.urls}}, \pageref{Contour:module-Contour.urls}
\end{theindex}

\renewcommand{\indexname}{Index}
\printindex
\end{document}
